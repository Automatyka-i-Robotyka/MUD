\documentclass{article}
\usepackage[utf8]{inputenc}
\usepackage{graphicx}
\usepackage{ragged2e}
\usepackage[margin=2.5cm]{geometry}
\usepackage{array}
\usepackage{wrapfig}
\usepackage{multirow}
\usepackage{tabularx}
\usepackage{amsmath}
\usepackage{wrapfig}
\usepackage{mathtools}
\usepackage[table]{xcolor}
\usepackage{multirow}
\usepackage{polski}
\usepackage{rotating}
\title{Sprawozdanie 1}

\author{Jan Bronicki 249011\\
Marcin Gruchała 248982\\}
\date{}
\begin{document}
\maketitle

\section{Cel ćwiczenia.}
Stworzenie charakterystyki statycznej dla pomieszczenia z grzejnikiem.
\section{Wstęp.}
Opisujemy model domu z grzejnikiem. Poniższy rysunek przedstawia model.
\begin{figure}[h]
    \centering
    \includegraphics[width=0.5\textwidth]{dom.png}
    \label{fig:my_label}
\end{figure}
%Zamienic kropki na przecinki


%%%%%%%%%%%%%%%%%%%%%%%%%%%%%%%%%%%%%%%%%%%%%%%%%%%%%%%%%%%%%
$\\
T_{zew} - \text{temperatura\ na\ zewnątrz\ pomieszczenia}\\
T_{wew} - \text{temperatura\ wewnątrz\ pomieszczenia} \\
T_{p} - \text{temperatura\ na\ poddaszu}\\
Q_g - \text{grzejnik}\\
K_{cw} - \text{współczynnik\ strat\ ciepła\ przez\ ściany} \\
K_{cwp} - \text{współczynnik\ strat\ ciepła\ przez\ sufit} \\
K_{cp} - \text{współczynnik\ strat\ ciepła\ przez\ dach}\\
$



%%%%%%%%%%%%%%%%%%%%%%%%%%%%%%%%%%%%%%%%%%%%%%%%%%%%%%%%%%%%%



% Wyprowadzenie wzorów oraz macierzy
\begin{center}
    $
    \begin{cases}
        0=Q_{g} -K_{cw} (T_{wew} - T_{zew})-K_{cwp} (T_{wew} -T_{p})\\
        0=K_{cwp}(T_{wew}-T_{p})-K_{cp}(T_{p}-T_{zew})
    \end{cases}
    $
    \vspace{1ex}
    $
    ,gdzie \ K_{cwp}=\alpha\cdot K_{cw}.\ \alpha=0,25
    $
    \vspace{1ex}
    $
    \begin{cases}
        K_{cw}T_{wew}-K_{cw}T_{zew}+0,25K_{cw}T_{zew}-0,25K_{cw}T_{p}=Q_{g}\\
        0,25K_{cw}T_{wew}-0,25K_{cw}T_{p}-K_{cp}T_{p}+K_{cp}T_{zew}=0
    \end{cases}
    $
    \vspace{1ex}
    $
        \begin{cases}
            K_{cw}T_{wew}+0,25K_{cw}T_{wew}-0,25K_{cw}T{p}=Q_{g}+K_{cw}T_{zew}\\
            0,25K_{cw}T_{wew}-0,25K_{cw}T_{p}-K_{cp}T_{p}=-0,25K_{cp}T_{zew}
        \end{cases}
    $
    \vspace{1ex}
    $
        \begin{cases}
            T_{wew}(K_{cw}+0,25K_{cw})-T_{p}(0,25K_{cw})=Q_{g}+K_{cw}T_{zew}\\
            T_{wew}(0,25K_{cw})-T_{p}(0,25K_{cw}-K_{cp})=-0,25K_{cp}T_{zew}
        \end{cases}
    $
    \vspace{1ex}\\
    $
    \begin{bmatrix}
        (K_{cw}+0,25K_{cw})& -(0,25K_{cw})            \\[0.3em]
        0,25K_{cw} & (0,25K_{cw}-K_{cp})            \\[0.3em]
    \end{bmatrix}
    \begin{bmatrix}
        T_{wew}\\
        T_{p}
    \end{bmatrix}
    =
    \begin{bmatrix}
        Q_{q}+K_{cw}T_{zew}\\
        -K_{cp}T_{zew}
    \end{bmatrix}
    $
    \vspace{1ex}
\end{center}
%%%%%%%%%%%%%%%%%%%%%%%%%%%%%%%%%%%%%%%%%%%%%%%%%%%%%%%%%%%%%

\newpage
\begin{flushleft}
\section{Rozwiązanie układów rwónań ze względu na $K_{cp}. K_{cwp}. K_{cw}$ \\oraz $T_{wew}. T_{p}$(tradycyjnie).}
\end{flushleft}
$$
 \begin{cases}
        0=Q_{g} -K_{cw} (T_{wew} - T_{zew})-K_{cwp} (T_{wew} -T_{p})\\
        0=K_{cwp}(T_{wew}-T_{p})-K_{cp}(T_{p}-T_{zew})
\end{cases}
$$

Ze względu na $K_{cp}. K_{cwp}. K_{cw}$:

Ze względu na  $T_{wew}. T_{p}$:

\begin{flushleft}
\section{Rozwiązanie układów rwónań ze względu na $K_{cp}. K_{cwp}. K_{cw}$ \\oraz $T_{wew}. T_{p}$(macierzowo).}
\end{flushleft}

Ze względu na $K_{cp}. K_{cwp}. K_{cw}$:
$$
\begin{bmatrix}
        0 & 1,25T_{wew}-0,25T_{p}-T_{zew}            \\[0.3em]
        T_{p}-T_{zew} & -0,25(T_{wew}-T_{p})            \\[0.3em]
    \end{bmatrix}
    \begin{bmatrix}
        K_{cp}\\
        K_{cw}
    \end{bmatrix}
    =
    \begin{bmatrix}
        Q_{q}\\
        0
    \end{bmatrix}
$$

$$
W=
\begin{bmatrix}
        0 & 1,25T_{wew}-0,25T_{p}-T_{zew}            \\[0.3em]
        T_{p}-T_{zew} & -0,25(T_{wew}-T_{p})            \\[0.3em]
    \end{bmatrix}
$$
$$
K=
\begin{bmatrix}
        K_{cp}\\
        K_{cw}
    \end{bmatrix}
$$

$$
Y=\begin{bmatrix}
        Q_{q}\\
        0
    \end{bmatrix}
$$

$$
WK=Y
$$

$$
K=W^{-1}Y=
\begin{bmatrix}
        1,960784313725490\\
        23,529411764705880
\end{bmatrix}\approx 
\begin{bmatrix}
        1,96 \\
        23,53
\end{bmatrix}
$$

\begin{flushleft}

Ze względu na  $T_{wew}. T_{p}$:

\end{flushleft}
$$
\begin{bmatrix}
        (K_{cw}+0,25K_{cw})& -(0,25K_{cw})            \\[0.3em]
        0,25K_{cw} & (0,25K_{cw}-K_{cp})            \\[0.3em]
    \end{bmatrix}
    \begin{bmatrix}
        T_{wew}\\
        T_{p}
    \end{bmatrix}
    =
    \begin{bmatrix}
        Q_{q}+K_{cw}T_{zew}\\
        -K_{cp}T_{zew}
    \end{bmatrix}
$$

$$
O=\begin{bmatrix}
        (K_{cw}+0,25K_{cw})& -(0,25K_{cw})            \\[0.3em]
        0,25K_{cw} & (0,25K_{cw}-K_{cp})            \\[0.3em]
    \end{bmatrix}
$$

$$
T=\begin{bmatrix}
        T_{wew}\\
        T_{p}
    \end{bmatrix}
$$

$$
P=\begin{bmatrix}
        Q_{q}+K_{cw}T_{zew}\\
        -K_{cp}T_{zew}
    \end{bmatrix}
$$

$$
OT=P
$$

$$
T=O^{-1}P=
\begin{bmatrix}
       20 \\
       10
\end{bmatrix}
$$

Macierze obliczyliśmy w matlabie.

\section{Wykresy.}
Wykresy wygenerowaliśmy w programie Matlab.

\begin{figure}
    \centering
    \begin{turn}{270}
    \includegraphics[width=0.6\textwidth]{Dom_wykresy.png}
    \end{turn}
    \label{fig:my_label}
\end{figure}

\section{Wnioski.}

---------------------------------------------------
%Dla K
\begin{center}
    $Q_{g}=1000W$ \\
    $T_{zew}=-20^{\circ}C$\\
    $T_{wew}=20^{\circ}C$\\
    $T_{p}=10^{\circ}C$\\
    $\alpha=0,25$\\
    $K_{cwp}=\alpha K_{cw} \implies K_{cwp}=0,25K_{cw}$\\
    \vspace{1ex}
    $
    \begin{cases}
        0=Q_{g} -K_{cw} (T_{wew} - T_{zew})-K_{cwp} (T_{wew} -T_{p})\\
        0=K_{cwp}(T_{wew}-T_{p})-K_{cp}(T_{p}-T_{zew})
    \end{cases}
    $
    \vspace{1ex}
    $
    \begin{cases}
        K_{cw}T_{wew}-K_{cw}T_{zew}+0,25K_{cw}T_{zew}-0,25K_{cw}T_{p}=Q_{g}\\
        0,25K_{cw}T_{wew}-0,25K_{cw}T_{p}-K_{cp}T_{p}+K_{cp}T_{zew}=0
    \end{cases}
    $
    \vspace{1ex}
    $
    \begin{cases}
        K_{cw}(1,25T_{wew}-0,25T_{p}-T_{zew})=Q_{q}\\
        K_{cp}(T_{p}-T_{zew})-0,25K_{cw}(T_{wew}-T_{p})=0
    \end{cases}
    $
    \vspace{1ex}
    \\
    $
    K_{cp}=\frac{0,25K_{cw}(T_{wew}-T_{p})}{T_{p}-T_{zew}}
    $
    \\
    \vspace{1ex}
    $
    K_{cw}=\frac{Q_{g}}{0,75T_{wew}-T_{zew}-0,25T_{p}}
    $
    \\
    \vspace{1ex}
    $
    K_{cw}=\frac{1000}{25-2,5+20}\approx 23,53
    $
    \\
    \vspace{1ex}
    $
    K_{cp}=\frac{0,25 \cdot 23,53(20-10)}{10-(-20)}\approx1,96
    $
    \\
    \vspace{1ex}
    $
    K_{cwp}=0,25K_{cw}\approx 5,98
    $

\vspace{1ex}
    $
    \begin{cases}
        0=Q_{g} -K_{cw} (T_{wew} - T_{zew})-K_{cwp} (T_{wew} -T_{p})\\
        0=K_{cwp}(T_{wew}-T_{p})-K_{cp}(T_{p}-T_{zew})
    \end{cases}
    $
    \\
    \vspace{1ex}
    $
    \begin{cases}
        T_{wew}=\frac{Q_{g}+K_{cw}T_{zew}+0,25T_{p}K_{cw}}{1,25K_{cw}}\\
        T_{p}=\frac{0,2Q_{g}+0,2T_{zew}K_{cw}+T_{zew}K_{cw}}{0,2K_{cw}+K_{cp}}
    \end{cases}
    $
    \\
    \vspace{1ex}
    $
    \begin{cases}
        T_{wew}=\frac{Q_{g}+K_{cw}T_{zew}+0,25K_{cw}\frac{0,2Q_{g}+0,2T_{zew}K_{cw}+T_{zew}K_{cw}}{0,2K_{cw}+K_{cp}}}{1,25K_{cw}}\\
        T_{p}=\frac{0,2Q_{g}+0,2T_{zew}K_{cw}+T_{zew}K_{cw}}{0,2K_{cw}+K_{cp}}
    \end{cases}
    $
    \\
    \vspace{1ex}
    $
    \begin{cases}
        T_{wew}=\frac{1000+23,53\cdot(-20)+0,25\cdot23,53\frac{0,2\cdot1000+0,2\cdot(-20)\cdot23,53+(-20)\cdot1,96}{0,2\cdot23,53+1,96}}{1,25\cdot23,53} \approx 20\\
        T_{p}=\frac{0,2\cdot1000+0,2\cdot(-20)\cdot23,53+(-20)\cdot1,96}{0,2\cdot23,53+1,96} \approx 10
    \end{cases}
    $
\end{center}




---------------------------------------------------

\end{document}